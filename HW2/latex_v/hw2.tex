\documentclass{article}
\usepackage{amsmath}
\usepackage{graphicx}
\usepackage{geometry}
\usepackage{xcolor} % for custom colors
\usepackage{float}       % for [H] placement
\usepackage{caption}     % for better caption control
\usepackage{subcaption} 
\usepackage{listings}

\lstset{
  language=Python,
  basicstyle=\ttfamily\small,
  keywordstyle=\color{blue}\bfseries,
  stringstyle=\color{red},
  commentstyle=\color{green!50!black},
  numbers=left,
  numberstyle=\tiny,
  stepnumber=1,
  numbersep=10pt,
  backgroundcolor=\color{gray!10},
  frame=single,
  breaklines=true,
  captionpos=b,
  tabsize=4,
  showspaces=false,
  showstringspaces=false
}

\geometry{a4paper, margin=1in}

\title{Homework 2}
\author{Your Name}
\date{\today}

\begin{document}

\maketitle


We will now start reflecting on the coding questions of Homework 2. The code base that can be used to replicate the result can be found
in the following link: \texttt{https://github.com/TagoreZhao/STAT260/tree/main/HW2}

\section{Problem 6, 7, and 8}

The code needed for generating these three matrices is given below:

\begin{lstlisting}[caption={Python code for generating matrices}]
import numpy as np

def generate_covariance_matrix(d):
    indices = np.arange(d)
    Sigma = 2 * 0.5 ** np.abs(indices[:, None] - indices[None, :])
    return Sigma

def generate_gaussian_A(n, d, seed=1234):
    rng = np.random.default_rng(seed)
    Sigma = generate_covariance_matrix(d)
    mean = np.ones(d)
    A = rng.multivariate_normal(mean, Sigma, size=n)
    return A

def generate_t_distribution_A(n, d, df, seed=1234):
    rng = np.random.default_rng(seed)
    Sigma = generate_covariance_matrix(d)
    mean = np.ones(d)
    z = rng.multivariate_normal(mean, Sigma, size=n)
    chi2_samples = rng.chisquare(df, size=(n, 1))
    A = z / np.sqrt(chi2_samples / df)
    return A
\end{lstlisting}
Since numpy does not provide built in functions for generating t-distributed random variables, we have to generate the random variables
ourselves. The Gaussian random variables are generated using the \texttt{multivariate\_normal}
function, while the t-distributed random variables are generated using the formula $A = Z / \sqrt{\chi^2 / df}$, where $Z$ is the Gaussian
random variable, $\chi^2$ is the chi-squared random variable, and $df$ is the degrees of freedom.
\newpage


\section*{Problem 9}

We will first plot the norm based probability distribution for all three matrices that we generated using seed 1234.

\begin{figure}[H]
    \centering
    \includegraphics[width=0.5\textwidth]{/home/tagore/repos/STAT260/HW2/assets/Q9_A_GA norm_based Sampling Probabilities.png}
    \caption{GA Norm based probability distribution}
    \label{fig:GA_norm_based_prob}
\end{figure}

\begin{figure}[H]
    \centering
    \includegraphics[width=0.5\textwidth]{/home/tagore/repos/STAT260/HW2/assets/Q9_A_T3 norm_based Sampling Probabilities.png}
    \caption{T3 Norm based probability distribution}
    \label{fig:T3_norm_based_prob}
\end{figure}

\begin{figure}[H]
    \centering
    \includegraphics[width=0.5\textwidth]{/home/tagore/repos/STAT260/HW2/assets/Q9_A_T1 norm_based Sampling Probabilities.png}
    \caption{T1 Norm based probability distribution}
    \label{fig:T1_norm_based_prob}
\end{figure}

We will now plot the Frobenius and spectral error for the approximations of the three matrix multiplications.

\begin{figure}[H]
    \centering
    \includegraphics[width=0.5\textwidth]{/home/tagore/repos/STAT260/HW2/assets/Q9_Error of Norm Based Sampling Approximation for (A_GA * A_GA^T).png}
    \caption{Error of Norm Based Sampling Approximation for (\(A_{GA}^\top A_{GA}\))}
    \label{fig:GA_norm_based_error}
\end{figure}

\begin{figure}[H]
    \centering
    \includegraphics[width=0.5\textwidth]{/home/tagore/repos/STAT260/HW2/assets/Q9_Error of Norm Based Sampling Approximation for (A_T3 * A_T3^T).png}
    \caption{Error of Norm Based Sampling Approximation for (\(A_{T3}^\top A_{T3}\))}
    \label{fig:T3s_norm_based_error}
\end{figure}

\begin{figure}[H]
    \centering
    \includegraphics[width=0.5\textwidth]{/home/tagore/repos/STAT260/HW2/assets/Q_9_Error of Norm Based Sampling Approximation for (A_T1 * A_T1^T).png}
    \caption{Error of Norm Based Sampling Approximation for (\(A_{T1}^\top A_{T1}\))}
    \label{fig:T1_norm_based_error}
\end{figure}

\newpage

\section*{Problem 10}

We will now plot the Frobenius and spectral error for the approximations of the three left singular matrices multiplication.

\begin{figure}[H]
    \centering
    \includegraphics[width=0.5\textwidth]{/home/tagore/repos/STAT260/HW2/assets/Q10_Uniform Sampling: Error vs. Samples for U_A_GA (U_A_GA * U_A_GA^T).png}
    \caption{Error of Uniform Based Sampling Approximation for (\(U_{GA}^\top U_{GA}\))}
    \label{fig:GA_uniform_based_error}
\end{figure}

\begin{figure}[H]
    \centering
    \includegraphics[width=0.5\textwidth]{/home/tagore/repos/STAT260/HW2/assets/Q10_Uniform Sampling: Error vs. Samples for U_A_T3 (U_A_T3 * U_A_T3^T).png}
    \caption{Error of Uniform Based Sampling Approximation for (\(U_{T3}^\top U_{T3}\))}
    \label{fig:T1_uniform_based_error}
\end{figure}

\begin{figure}[H]
    \centering
    \includegraphics[width=0.5\textwidth]{/home/tagore/repos/STAT260/HW2/assets/Q10_Uniform Sampling: Error vs. Samples for U_A_T1 (U_A_T1 * U_A_T1^T).png}
    \caption{Error of Uniform Based Sampling Approximation for (\(U_{T1}^\top U_{T1}\))}
    \label{fig:T3_uniform_based_error}
\end{figure}

\newpage

\section*{Problem 11}

We will now plot the Frobenius and spectral error for the approximations of \(A^\top A\) using leverage score sampling.

\begin{figure}[H]
    \centering
    \includegraphics[width=0.5\textwidth]{/home/tagore/repos/STAT260/HW2/assets/Q11_Leverage Sampling: Error vs. Samples for A_GA (A_GA * A_GA^T).png}
    \caption{Error of Leverage Based Sampling Approximation for (\(A_{GA}^\top A_{GA}\))}
    \label{fig:GA_leverage_based_error}
\end{figure}

\begin{figure}[H]
    \centering
    \includegraphics[width=0.5\textwidth]{/home/tagore/repos/STAT260/HW2/assets/Q11_Leverage Sampling: Error vs. Samples for A_T3 (A_T3 * A_T3^T).png}
    \caption{Error of Leverage Based Sampling Approximation for (\(A_{T3}^\top A_{T3}\))}
    \label{fig:T1_leverage_based_error}
\end{figure}

\begin{figure}[H]
    \centering
    \includegraphics[width=0.5\textwidth]{/home/tagore/repos/STAT260/HW2/assets/Q11_Leverage Sampling: Error vs. Samples for A_T1 (A_T1 * A_T1^T).png}
    \caption{Error of Leverage Based Sampling Approximation for (\(A_{T1}^\top A_{T1}\))}
    \label{fig:T3_leverage_based_error}
\end{figure}

The results looks similar for $A_{GA}$ and $A_{T3}$, but the error for $A_{T1}$ is significantly higher than the other two matrices. 
This means that the leverage score sampling is not as effective for $A_{T1}$ as it is for the other two matrices.


\section*{Problem 12}

We will now plot the Frobenius and spectral error for the approximations of \(A A^\top\) using gaussian projection and $\{\pm 1\}$ projection.
\begin{figure}[H]
    \centering
    \includegraphics[width=0.5\textwidth]{/home/tagore/repos/STAT260/HW2/assets/Q12_Gaussian Projection: Error vs. Samples for A_GA (A_GA * A_GA^T).png}
    \caption{Error of Gaussian Projection Approximation for (\(A_{GA} A_{GA}^\top\))}
    \label{fig:GA_gaussian_projection_error}
\end{figure}
\begin{figure}[H]
    \centering
    \includegraphics[width=0.5\textwidth]{/home/tagore/repos/STAT260/HW2/assets/Q12_Gaussian Projection: Error vs. Samples for A_T3 (A_T3 * A_T3^T).png}
    \caption{Error of Gaussian Projection Approximation for (\(A_{T3} A_{T3}^\top\))}
    \label{fig:T3_gaussian_projection_error}
\end{figure}

\begin{figure}[H]
    \centering
    \includegraphics[width=0.5\textwidth]{/home/tagore/repos/STAT260/HW2/assets/Q12_Gaussian Projection: Error vs. Samples for A_T1 (A_T1 * A_T1^T).png}
    \caption{Error of Gaussian Projection Approximation for (\(A_{T1} A_{T1}^\top\))}
    \label{fig:T1_gaussian_projection_error}
\end{figure}

\begin{figure}[H]
    \centering
    \includegraphics[width=0.5\textwidth]{/home/tagore/repos/STAT260/HW2/assets/Q12_Sign Projection: Error vs. Samples for A_GA (A_GA * A_GA^T).png}
    \caption{Error of $\{\pm 1\}$ Projection Approximation for (\(A_{GA} A_{GA}^\top\))}
    \label{fig:GA_pm1_projection_error}
\end{figure}

\begin{figure}[H]
    \centering
    \includegraphics[width=0.5\textwidth]{/home/tagore/repos/STAT260/HW2/assets/Q12_Sign Projection: Error vs. Samples for A_T3 (A_T3 * A_T3^T).png}
    \caption{Error of $\{\pm 1\}$ Projection Approximation for (\(A_{T3} A_{T3}^\top\))}
    \label{fig:T3_pm1_projection_error}
\end{figure}\

\begin{figure}[H]
    \centering
    \includegraphics[width=0.5\textwidth]{/home/tagore/repos/STAT260/HW2/assets/Q12_Sign Projection: Error vs. Samples for A_T1 (A_T1 * A_T1^T).png}
    \caption{Error of $\{\pm 1\}$ Projection Approximation for (\(A_{T1} A_{T1}^\top\))}
    \label{fig:T1_pm1_projection_error}
\end{figure}

Comparing the results of the Gaussian projection and $\{\pm 1\}$ projection: The gaussian projection produces more stable results and has slightly smaller error for projecting $A_{GA}$ and $A_{T3}$ when there is reasonbale amount of dimensions. 
The guassian projection performs extremelly well for $A_{T1}$, while the $\{\pm 1\}$ projection has a higher error for all three matrices. The only extreme case is that sign projection seems to outperform gaussian projection for $A_{T1}$ when the number of dimensions is small.

Comparing the results of projection and sampling: The projection method has much lower error compare to uniform sampling and overall slight lower error when comparing to norm based sampling and leverage score sampling. The projection method is more stable and has lower error for all three matrices.
However, there are times where the projection method shows outlier errors, such as the gaussian projection for $A_{GA}$ when the number of dimensions is small.

\section*{Problem 13}

We will now plot 3D plot of the error of the sparse approximations of \(A A^\top\) using gaussian projection and $\{\pm 1\}$ projection. The two axes are the sparsity and the number of dimensions, and the third axis is the error.

\begin{figure}[H]
    \centering
    \includegraphics[width=1\textwidth]{/home/tagore/repos/STAT260/HW2/assets/Q_13_A_GA_gaussian_errors.png}
    \caption{Error of Gaussian Projection Approximation for (\(A_{GA} A_{GA}^\top\))}
    \label{fig:GA_gaussian_projection_error_3d}
\end{figure}

\begin{figure}[H]
    \centering
    \includegraphics[width=1\textwidth]{/home/tagore/repos/STAT260/HW2/assets/Q_13_A_GA_sign_errors.png}
    \caption{Error of $\{\pm 1\}$ Projection Approximation for (\(A_{GA} A_{GA}^\top\))}
    \label{fig:GA_pm1_projection_error_3d}
\end{figure}

\begin{figure}[H]
    \centering
    \includegraphics[width=1\textwidth]{/home/tagore/repos/STAT260/HW2/assets/Q_13_A_T3_gaussian_errors.png}
    \caption{Error of Gaussian Projection Approximation for (\(A_{T3} A_{T3}^\top\))}
    \label{fig:T3_gaussian_projection_error_3d}
\end{figure}

\begin{figure}[H]
    \centering
    \includegraphics[width=1\textwidth]{/home/tagore/repos/STAT260/HW2/assets/Q_13_A_T3_sign_errors.png}
    \caption{Error of $\{\pm 1\}$ Projection Approximation for (\(A_{T3} A_{T3}^\top\))}
    \label{fig:T3_pm1_projection_error_3d}
\end{figure}

\begin{figure}[H]
    \centering
    \includegraphics[width=1\textwidth]{/home/tagore/repos/STAT260/HW2/assets/Q_13_A_T1_gaussian_errors.png}
    \caption{Error of Gaussian Projection Approximation for (\(A_{T1} A_{T1}^\top\))}
    \label{fig:T1_gaussian_projection_error_3d}
\end{figure}
 
\begin{figure}[H]
    \centering
    \includegraphics[width=1\textwidth]{/home/tagore/repos/STAT260/HW2/assets/Q_13_A_T1_sign_errors.png}
    \caption{Error of $\{\pm 1\}$ Projection Approximation for (\(A_{T1} A_{T1}^\top\))}
    \label{fig:T1_pm1_projection_error_3d}
\end{figure}

Both Gaussian projection and $\{\pm 1\}$ projection have similar error patterns for all three matrices. For $A_{GA}$ and $A_{T3}$, the error is relatively stable and low when the number of dimensions is large. 
In addition, both projection seems to perform well as we increase sparsity. However, for $A_{T1}$, the error is significantly higher than the other two matrices, and the error is not as stable as the other two matrices.
For both projection, the errors are higher when the sparsity is high, and the errors are more unstable when the sparsity is high.


\section*{Problem 13.5 Variability Analysis}
\subsection*{\(A_{GA}\)}

\begin{figure}[H]
    \centering
    % Frobenius Error Comparison
    \begin{subfigure}[b]{0.48\textwidth}
        \centering
        \includegraphics[width=\textwidth]{/home/tagore/repos/STAT260/HW2/assets/Q13.5_A_GA_frobenius_error.png}
        \caption{Sampling: Frobenius Error}
        \label{fig:GA_fs}
    \end{subfigure}
    \hfill
    \begin{subfigure}[b]{0.48\textwidth}
        \centering
        \includegraphics[width=\textwidth]{/home/tagore/repos/STAT260/HW2/assets/Q13.5_A_GA_projection_frobenius_error.png}
        \caption{Projection: Frobenius Error}
        \label{fig:GA_fp}
    \end{subfigure}
    
    \vspace{0.5cm}
    
    % Spectral Error Comparison
    \begin{subfigure}[b]{0.48\textwidth}
        \centering
        \includegraphics[width=\textwidth]{/home/tagore/repos/STAT260/HW2/assets/Q13.5_A_GA_spectral_error.png}
        \caption{Sampling: Spectral Error}
        \label{fig:GA_ss}
    \end{subfigure}
    \hfill
    \begin{subfigure}[b]{0.48\textwidth}
        \centering
        \includegraphics[width=\textwidth]{/home/tagore/repos/STAT260/HW2/assets/Q13.5_A_GA_projection_spectral_error.png}
        \caption{Projection: Spectral Error}
        \label{fig:GA_sp}
    \end{subfigure}
    
    \caption{\(A_{GA}\): Comparison of Random Sampling and Random Projection}
    \label{fig:GA_comparison}
\end{figure}

\newpage
\subsection*{\(A_{T1}\)}

\begin{figure}[H]
    \centering
    % Frobenius Error Comparison
    \begin{subfigure}[b]{0.48\textwidth}
        \centering
        \includegraphics[width=\textwidth]{/home/tagore/repos/STAT260/HW2/assets/Q13.5_A_T1_frobenius_error.png}
        \caption{Sampling: Frobenius Error}
        \label{fig:T1_fs}
    \end{subfigure}
    \hfill
    \begin{subfigure}[b]{0.48\textwidth}
        \centering
        \includegraphics[width=\textwidth]{/home/tagore/repos/STAT260/HW2/assets/Q13.5_A_T1_projection_frobenius_error.png}
        \caption{Projection: Frobenius Error}
        \label{fig:T1_fp}
    \end{subfigure}
    
    \vspace{0.5cm}
    
    % Spectral Error Comparison
    \begin{subfigure}[b]{0.48\textwidth}
        \centering
        \includegraphics[width=\textwidth]{/home/tagore/repos/STAT260/HW2/assets/Q13.5_A_T1_spectral_error.png}
        \caption{Sampling: Spectral Error}
        \label{fig:T1_ss}
    \end{subfigure}
    \hfill
    \begin{subfigure}[b]{0.48\textwidth}
        \centering
        \includegraphics[width=\textwidth]{/home/tagore/repos/STAT260/HW2/assets/Q13.5_A_T1_projection_spectral_error.png}
        \caption{Projection: Spectral Error}
        \label{fig:T1_sp}
    \end{subfigure}
    
    \caption{\(A_{T1}\): Comparison of Random Sampling and Random Projection}
    \label{fig:T1_comparison}
\end{figure}

\newpage
\subsection*{\(A_{T3}\)}

\begin{figure}[H]
    \centering
    % Frobenius Error Comparison
    \begin{subfigure}[b]{0.48\textwidth}
        \centering
        \includegraphics[width=\textwidth]{/home/tagore/repos/STAT260/HW2/assets/Q13.5_A_T3_frobenius_error.png}
        \caption{Sampling: Frobenius Error}
        \label{fig:T3_fs}
    \end{subfigure}
    \hfill
    \begin{subfigure}[b]{0.48\textwidth}
        \centering
        \includegraphics[width=\textwidth]{/home/tagore/repos/STAT260/HW2/assets/Q13.5_A_T3_projection_frobenius_error.png}
        \caption{Projection: Frobenius Error}
        \label{fig:T3_fp}
    \end{subfigure}
    
    \vspace{0.5cm}
    
    % Spectral Error Comparison
    \begin{subfigure}[b]{0.48\textwidth}
        \centering
        \includegraphics[width=\textwidth]{/home/tagore/repos/STAT260/HW2/assets/Q13.5_A_T3_spectral_error.png}
        \caption{Sampling: Spectral Error}
        \label{fig:T3_ss}
    \end{subfigure}
    \hfill
    \begin{subfigure}[b]{0.48\textwidth}
        \centering
        \includegraphics[width=\textwidth]{/home/tagore/repos/STAT260/HW2/assets/Q13.5_A_T3_projection_spectral_error.png}
        \caption{Projection: Spectral Error}
        \label{fig:T3_sp}
    \end{subfigure}
    
    \caption{\(A_{T3}\): Comparison of Random Sampling and Random Projection}
    \label{fig:T3_comparison}
\end{figure}


\section*{Problem 17}

For each matrix, the following figures show the LS error versus the number of samples \(r\) using three sampling methods. The top row corresponds to Regime 1 (r = d ... 2d) and the bottom row to Regime 2 (r = 2d, 3d, \(\ldots\)).

\subsection*{\(A_{GA}\)}

\begin{figure}[H]
    \centering
    % Regime 1
    \begin{subfigure}[b]{0.32\textwidth}
        \centering
        \includegraphics[width=\textwidth]{/home/tagore/repos/STAT260/HW2/assets/Q17_A_GA_uniform_regime1_LS_Error_vs_r.png}
        \caption{Uniform, Regime 1}
        \label{fig:GA_uniform_regime1}
    \end{subfigure}
    \begin{subfigure}[b]{0.32\textwidth}
        \centering
        \includegraphics[width=\textwidth]{/home/tagore/repos/STAT260/HW2/assets/Q17_A_GA_norm_based_regime1_LS_Error_vs_r.png}
        \caption{Norm-based, Regime 1}
        \label{fig:GA_norm_based_regime1}
    \end{subfigure}
    \begin{subfigure}[b]{0.32\textwidth}
        \centering
        \includegraphics[width=\textwidth]{/home/tagore/repos/STAT260/HW2/assets/Q17_A_GA_leverage_regime1_LS_Error_vs_r.png}
        \caption{Leverage, Regime 1}
        \label{fig:GA_leverage_regime1}
    \end{subfigure}
    
    \vspace{0.5cm}
    
    % Regime 2
    \begin{subfigure}[b]{0.32\textwidth}
        \centering
        \includegraphics[width=\textwidth]{/home/tagore/repos/STAT260/HW2/assets/Q17_A_GA_uniform_regime2_LS_Error_vs_r.png}
        \caption{Uniform, Regime 2}
        \label{fig:GA_uniform_regime2}
    \end{subfigure}
    \begin{subfigure}[b]{0.32\textwidth}
        \centering
        \includegraphics[width=\textwidth]{/home/tagore/repos/STAT260/HW2/assets/Q17_A_GA_norm_based_regime2_LS_Error_vs_r.png}
        \caption{Norm-based, Regime 2}
        \label{fig:GA_norm_based_regime2}
    \end{subfigure}
    \begin{subfigure}[b]{0.32\textwidth}
        \centering
        \includegraphics[width=\textwidth]{/home/tagore/repos/STAT260/HW2/assets/Q17_A_GA_leverage_regime2_LS_Error_vs_r.png}
        \caption{Leverage, Regime 2}
        \label{fig:GA_leverage_regime2}
    \end{subfigure}
    
    \caption{\(A_{GA}\): LS Error vs. r for Different Sampling Methods and Regimes}
    \label{fig:GA_comparison}
\end{figure}

\newpage

\subsection*{\(A_{T1}\)}

\begin{figure}[H]
    \centering
    % Regime 1
    \begin{subfigure}[b]{0.32\textwidth}
        \centering
        \includegraphics[width=\textwidth]{/home/tagore/repos/STAT260/HW2/assets/Q17_A_T1_uniform_regime1_LS_Error_vs_r.png}
        \caption{Uniform, Regime 1}
        \label{fig:T1_uniform_regime1}
    \end{subfigure}
    \begin{subfigure}[b]{0.32\textwidth}
        \centering
        \includegraphics[width=\textwidth]{/home/tagore/repos/STAT260/HW2/assets/Q17_A_T1_norm_based_regime1_LS_Error_vs_r.png}
        \caption{Norm-based, Regime 1}
        \label{fig:T1_norm_based_regime1}
    \end{subfigure}
    \begin{subfigure}[b]{0.32\textwidth}
        \centering
        \includegraphics[width=\textwidth]{/home/tagore/repos/STAT260/HW2/assets/Q17_A_T1_leverage_regime1_LS_Error_vs_r.png}
        \caption{Leverage, Regime 1}
        \label{fig:T1_leverage_regime1}
    \end{subfigure}
    
    \vspace{0.5cm}
    
    % Regime 2
    \begin{subfigure}[b]{0.32\textwidth}
        \centering
        \includegraphics[width=\textwidth]{/home/tagore/repos/STAT260/HW2/assets/Q17_A_T1_uniform_regime2_LS_Error_vs_r.png}
        \caption{Uniform, Regime 2}
        \label{fig:T1_uniform_regime2}
    \end{subfigure}
    \begin{subfigure}[b]{0.32\textwidth}
        \centering
        \includegraphics[width=\textwidth]{/home/tagore/repos/STAT260/HW2/assets/Q17_A_T1_norm_based_regime2_LS_Error_vs_r.png}
        \caption{Norm-based, Regime 2}
        \label{fig:T1_norm_based_regime2}
    \end{subfigure}
    \begin{subfigure}[b]{0.32\textwidth}
        \centering
        \includegraphics[width=\textwidth]{/home/tagore/repos/STAT260/HW2/assets/Q17_A_T1_leverage_regime2_LS_Error_vs_r.png}
        \caption{Leverage, Regime 2}
        \label{fig:T1_leverage_regime2}
    \end{subfigure}
    
    \caption{\(A_{T1}\): LS Error vs. r for Different Sampling Methods and Regimes}
    \label{fig:T1_comparison}
\end{figure}

\newpage

\subsection*{\(A_{T3}\)}

\begin{figure}[H]
    \centering
    % Regime 1
    \begin{subfigure}[b]{0.32\textwidth}
        \centering
        \includegraphics[width=\textwidth]{/home/tagore/repos/STAT260/HW2/assets/Q17_A_T3_uniform_regime1_LS_Error_vs_r.png}
        \caption{Uniform, Regime 1}
        \label{fig:T3_uniform_regime1}
    \end{subfigure}
    \begin{subfigure}[b]{0.32\textwidth}
        \centering
        \includegraphics[width=\textwidth]{/home/tagore/repos/STAT260/HW2/assets/Q17_A_T3_norm_based_regime1_LS_Error_vs_r.png}
        \caption{Norm-based, Regime 1}
        \label{fig:T3_norm_based_regime1}
    \end{subfigure}
    \begin{subfigure}[b]{0.32\textwidth}
        \centering
        \includegraphics[width=\textwidth]{/home/tagore/repos/STAT260/HW2/assets/Q17_A_T3_leverage_regime1_LS_Error_vs_r.png}
        \caption{Leverage, Regime 1}
        \label{fig:T3_leverage_regime1}
    \end{subfigure}
    
    \vspace{0.5cm}
    
    % Regime 2
    \begin{subfigure}[b]{0.32\textwidth}
        \centering
        \includegraphics[width=\textwidth]{/home/tagore/repos/STAT260/HW2/assets/Q17_A_T3_uniform_regime2_LS_Error_vs_r.png}
        \caption{Uniform, Regime 2}
        \label{fig:T3_uniform_regime2}
    \end{subfigure}
    \begin{subfigure}[b]{0.32\textwidth}
        \centering
        \includegraphics[width=\textwidth]{/home/tagore/repos/STAT260/HW2/assets/Q17_A_T3_norm_based_regime2_LS_Error_vs_r.png}
        \caption{Norm-based, Regime 2}
        \label{fig:T3_norm_based_regime2}
    \end{subfigure}
    \begin{subfigure}[b]{0.32\textwidth}
        \centering
        \includegraphics[width=\textwidth]{/home/tagore/repos/STAT260/HW2/assets/Q17_A_T3_leverage_regime2_LS_Error_vs_r.png}
        \caption{Leverage, Regime 2}
        \label{fig:T3_leverage_regime2}
    \end{subfigure}
    
    \caption{\(A_{T3}\): LS Error vs. r for Different Sampling Methods and Regimes}
    \label{fig:T3_comparison}
\end{figure}

\section*{Problem 18}


For each matrix, the figures below compare the LS error versus the number of samples \(r\) using two projection methods: Gaussian and Sign. The top row corresponds to Regime 1 (r = d ... 2d) and the bottom row to Regime 2 (r = 2d, 3d, \(\ldots\)).

\subsection*{\(A_{GA}\)}

\begin{figure}[H]
    \centering
    % Regime 1: Gaussian and Sign
    \begin{subfigure}[b]{0.48\textwidth}
        \centering
        \includegraphics[width=\textwidth]{/home/tagore/repos/STAT260/HW2/assets/Q18_A_GA_gaussian_regime1_Projection_LS_Error_vs_r.png}
        \caption{Gaussian, Regime 1}
        \label{fig:GA_gaussian_regime1}
    \end{subfigure}
    \hfill
    \begin{subfigure}[b]{0.48\textwidth}
        \centering
        \includegraphics[width=\textwidth]{/home/tagore/repos/STAT260/HW2/assets/Q18_A_GA_sign_regime1_Projection_LS_Error_vs_r.png}
        \caption{Sign, Regime 1}
        \label{fig:GA_sign_regime1}
    \end{subfigure}
    
    \vspace{0.5cm}
    
    % Regime 2: Gaussian and Sign
    \begin{subfigure}[b]{0.48\textwidth}
        \centering
        \includegraphics[width=\textwidth]{/home/tagore/repos/STAT260/HW2/assets/Q18_A_GA_gaussian_regime2_Projection_LS_Error_vs_r.png}
        \caption{Gaussian, Regime 2}
        \label{fig:GA_gaussian_regime2}
    \end{subfigure}
    \hfill
    \begin{subfigure}[b]{0.48\textwidth}
        \centering
        \includegraphics[width=\textwidth]{/home/tagore/repos/STAT260/HW2/assets/Q18_A_GA_sign_regime2_Projection_LS_Error_vs_r.png}
        \caption{Sign, Regime 2}
        \label{fig:GA_sign_regime2}
    \end{subfigure}
    
    \caption{\(A_{GA}\): Projection LS Error vs. \(r\) for Gaussian and Sign methods}
    \label{fig:GA_projection_comparison}
\end{figure}

\newpage

\subsection*{\(A_{T1}\)}

\begin{figure}[H]
    \centering
    % Regime 1: Gaussian and Sign
    \begin{subfigure}[b]{0.48\textwidth}
        \centering
        \includegraphics[width=\textwidth]{/home/tagore/repos/STAT260/HW2/assets/Q18_A_T1_gaussian_regime1_Projection_LS_Error_vs_r.png}
        \caption{Gaussian, Regime 1}
        \label{fig:T1_gaussian_regime1}
    \end{subfigure}
    \hfill
    \begin{subfigure}[b]{0.48\textwidth}
        \centering
        \includegraphics[width=\textwidth]{/home/tagore/repos/STAT260/HW2/assets/Q18_A_T1_sign_regime1_Projection_LS_Error_vs_r.png}
        \caption{Sign, Regime 1}
        \label{fig:T1_sign_regime1}
    \end{subfigure}
    
    \vspace{0.5cm}
    
    % Regime 2: Gaussian and Sign
    \begin{subfigure}[b]{0.48\textwidth}
        \centering
        \includegraphics[width=\textwidth]{/home/tagore/repos/STAT260/HW2/assets/Q18_A_T1_gaussian_regime2_Projection_LS_Error_vs_r.png}
        \caption{Gaussian, Regime 2}
        \label{fig:T1_gaussian_regime2}
    \end{subfigure}
    \hfill
    \begin{subfigure}[b]{0.48\textwidth}
        \centering
        \includegraphics[width=\textwidth]{/home/tagore/repos/STAT260/HW2/assets/Q18_A_T1_sign_regime2_Projection_LS_Error_vs_r.png}
        \caption{Sign, Regime 2}
        \label{fig:T1_sign_regime2}
    \end{subfigure}
    
    \caption{\(A_{T1}\): Projection LS Error vs. \(r\) for Gaussian and Sign methods}
    \label{fig:T1_projection_comparison}
\end{figure}

\newpage

\subsection*{\(A_{T3}\)}

\begin{figure}[H]
    \centering
    % Regime 1: Gaussian and Sign
    \begin{subfigure}[b]{0.48\textwidth}
        \centering
        \includegraphics[width=\textwidth]{/home/tagore/repos/STAT260/HW2/assets/Q18_A_T3_gaussian_regime1_Projection_LS_Error_vs_r.png}
        \caption{Gaussian, Regime 1}
        \label{fig:T3_gaussian_regime1}
    \end{subfigure}
    \hfill
    \begin{subfigure}[b]{0.48\textwidth}
        \centering
        \includegraphics[width=\textwidth]{/home/tagore/repos/STAT260/HW2/assets/Q18_A_T3_sign_regime1_Projection_LS_Error_vs_r.png}
        \caption{Sign, Regime 1}
        \label{fig:T3_sign_regime1}
    \end{subfigure}
    
    \vspace{0.5cm}
    
    % Regime 2: Gaussian and Sign
    \begin{subfigure}[b]{0.48\textwidth}
        \centering
        \includegraphics[width=\textwidth]{/home/tagore/repos/STAT260/HW2/assets/Q18_A_T3_gaussian_regime2_Projection_LS_Error_vs_r.png}
        \caption{Gaussian, Regime 2}
        \label{fig:T3_gaussian_regime2}
    \end{subfigure}
    \hfill
    \begin{subfigure}[b]{0.48\textwidth}
        \centering
        \includegraphics[width=\textwidth]{/home/tagore/repos/STAT260/HW2/assets/Q18_A_T3_sign_regime2_Projection_LS_Error_vs_r.png}
        \caption{Sign, Regime 2}
        \label{fig:T3_sign_regime2}
    \end{subfigure}
    
    \caption{\(A_{T3}\): Projection LS Error vs. \(r\) for Gaussian and Sign methods}
    \label{fig:T3_projection_comparison}
\end{figure}
\section*{Problem 19}
In this problem, we compare the projection LS error for different projection methods on three matrices. For each matrix, the figures below display the results using Gaussian projection and Sign projection.

\subsection*{\(A_{GA}\)}
\begin{figure}[H]
    \centering
    \begin{subfigure}[b]{0.48\textwidth}
        \centering
        \includegraphics[width=\textwidth]{/home/tagore/repos/STAT260/HW2/assets/Q19_A_GA_gaussian_projection_LS_error.png}
        \caption{Gaussian Projection}
        \label{fig:Q19_AGA_gaussian}
    \end{subfigure}
    \hfill
    \begin{subfigure}[b]{0.48\textwidth}
        \centering
        \includegraphics[width=\textwidth]{/home/tagore/repos/STAT260/HW2/assets/Q19_A_GA_sign_projection_LS_error.png}
        \caption{Sign Projection}
        \label{fig:Q19_AGA_sign}
    \end{subfigure}
    \caption{\(A_{GA}\): Projection LS Error Comparison}
    \label{fig:Q19_AGA}
\end{figure}

\newpage
\subsection*{\(A_{T1}\)}
\begin{figure}[H]
    \centering
    \begin{subfigure}[b]{0.48\textwidth}
        \centering
        \includegraphics[width=\textwidth]{/home/tagore/repos/STAT260/HW2/assets/Q19_A_T1_gaussian_projection_LS_error.png}
        \caption{Gaussian Projection}
        \label{fig:Q19_AT1_gaussian}
    \end{subfigure}
    \hfill
    \begin{subfigure}[b]{0.48\textwidth}
        \centering
        \includegraphics[width=\textwidth]{/home/tagore/repos/STAT260/HW2/assets/Q19_A_T1_sign_projection_LS_error.png}
        \caption{Sign Projection}
        \label{fig:Q19_AT1_sign}
    \end{subfigure}
    \caption{\(A_{T1}\): Projection LS Error Comparison}
    \label{fig:Q19_AT1}
\end{figure}

\newpage
\subsection*{\(A_{T3}\)}
\begin{figure}[H]
    \centering
    \begin{subfigure}[b]{0.48\textwidth}
        \centering
        \includegraphics[width=\textwidth]{/home/tagore/repos/STAT260/HW2/assets/Q19_A_T3_gaussian_projection_LS_error.png}
        \caption{Gaussian Projection}
        \label{fig:Q19_AT3_gaussian}
    \end{subfigure}
    \hfill
    \begin{subfigure}[b]{0.48\textwidth}
        \centering
        \includegraphics[width=\textwidth]{/home/tagore/repos/STAT260/HW2/assets/Q19_A_T3_sign_projection_LS_error.png}
        \caption{Sign Projection}
        \label{fig:Q19_AT3_sign}
    \end{subfigure}
    \caption{\(A_{T3}\): Projection LS Error Comparison}
    \label{fig:Q19_AT3}
\end{figure}


\section*{Problem 20: LS Error and Running Time Comparison}

Figure~\ref{fig:running_time} shows the running time of the LS problem solved by three methods:
\begin{itemize}
    \item \textbf{Random Projection LS:} Uses a dense random projection (with entries $\pm1$) to reduce the size of the problem before solving it.
    \item \textbf{QR-based LS:} Solves the full problem by computing a QR decomposition.
    \item \textbf{SVD-based LS:} Solves the full problem via singular value decomposition.
\end{itemize}

For smaller problem sizes (i.e., smaller $n$), the overhead of generating and applying a dense random projection matrix (since we are not using a fast Hadamard-based transform) makes the random projection method comparable or even slightly slower than the highly optimized QR and SVD routines. However, for larger problems, reducing the dimension from $n$ to $r$ (with $r\ll n$) significantly reduces the computational burden of solving the LS problem, and the random projection method becomes slightly faster and can handle larger problems than QR or SVD.

\begin{figure}[H]
    \centering
    \includegraphics[width=0.8\textwidth]{/home/tagore/repos/STAT260/HW2/assets/Q20_running_time_comparison.png}
    \caption{Running Time Comparison of LS Methods for $d=500$. For small $n$, the overhead of random projection leads to comparable running times, while for large $n$, the projection method becomes slightly faster than QR or SVD.}
    \label{fig:running_time}
\end{figure}
\end{document}