\documentclass{article}
\usepackage{amsmath}
\usepackage{graphicx}
\usepackage{geometry}
\usepackage{xcolor} % for custom colors
\usepackage{listings}

\lstset{
  language=Python,
  basicstyle=\ttfamily\small,
  keywordstyle=\color{blue}\bfseries,
  stringstyle=\color{red},
  commentstyle=\color{green!50!black},
  numbers=left,
  numberstyle=\tiny,
  stepnumber=1,
  numbersep=10pt,
  backgroundcolor=\color{gray!10},
  frame=single,
  breaklines=true,
  captionpos=b,
  tabsize=4,
  showspaces=false,
  showstringspaces=false
}

\geometry{a4paper, margin=1in}

\title{Homework 2}
\author{Your Name}
\date{\today}

\begin{document}

\maketitle


We will now start reflecting on the coding questions of Homework 2. The code base that can be used to replicate the result can be found
in the following link: \texttt{https://github.com/TagoreZhao/STAT260/tree/main/HW2}

\section{Problem 6, 7, and 8}

The code needed for generating these three matrices is given below:

\begin{lstlisting}[caption={Python code for generating matrices}]
import numpy as np

def generate_covariance_matrix(d):
    indices = np.arange(d)
    Sigma = 2 * 0.5 ** np.abs(indices[:, None] - indices[None, :])
    return Sigma

def generate_gaussian_A(n, d, seed=1234):
    rng = np.random.default_rng(seed)
    Sigma = generate_covariance_matrix(d)
    mean = np.ones(d)
    A = rng.multivariate_normal(mean, Sigma, size=n)
    return A

def generate_t_distribution_A(n, d, df, seed=1234):
    rng = np.random.default_rng(seed)
    Sigma = generate_covariance_matrix(d)
    mean = np.ones(d)
    z = rng.multivariate_normal(mean, Sigma, size=n)
    chi2_samples = rng.chisquare(df, size=(n, 1))
    A = z / np.sqrt(chi2_samples / df)
    return A
\end{lstlisting}
Since numpy does not provide built in functions for generating t-distributed random variables, we have to generate the random variables
ourselves. The Gaussian random variables are generated using the \texttt{multivariate\_normal}
function, while the t-distributed random variables are generated using the formula $A = Z / \sqrt{\chi^2 / df}$, where $Z$ is the Gaussian
random variable, $\chi^2$ is the chi-squared random variable, and $df$ is the degrees of freedom.

\section*{Problem 9}
We 


% \section*{Problem 2}
% Here is an example of a graph:
% \begin{figure}[h!]
%     \centering
%     \includegraphics[width=0.5\textwidth]{example-graph.png}
%     \caption{Example Graph}
%     \label{fig:example-graph}
% \end{figure}

\end{document}